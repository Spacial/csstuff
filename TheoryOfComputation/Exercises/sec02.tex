\section{Exemplos de Gramáticas e Linguagnes}

\begin{itemize}
  \item Gramáticas são usadas para \underline{gerar} linguagens.
  \item Relação entre gramáticas e linguagens:
   \begin{itemize}
     \item[$\rightarrow$] Especificar uma linguagem e construir uma gramática que a gere.
     \item[$\rightarrow$] Dada uma gramática, qual a linguagem gerada por ela?
   \end{itemize}
  \item É preciso ter experiência para fazer estas duas tarefas!
\end{itemize}

\subsection{Exemplos}

\textbf{\underline{Exemplo 1}}:\\
   \[
%     \begin{Huge} \{ \end{Huge} 
%     \begin{tabular}{l c l}
      P:\left\{
      \begin{array}{l c l}
        S & \rightarrow & \ \texttt{a}S\texttt{a} \quad | \quad \texttt{a}B\texttt{a} \\
	B & \rightarrow & \ \texttt{b}B  \quad | \quad \texttt{b}  \\
      \end{array}
      \right.
  \]
\begin{center}
  {\color{red} $ L(G) = ${\Huge ?} }
  \makebox[\linewidth]{\rule{\paperwidth/2}{0.4pt}}
  \vspace{1cm}
  {\color{blue} $ L(G) = \{ a^{n}b^{m}a^{n} \quad | \quad n \ > \ 0, \ m \ > \ 0 \}$ }
\end{center}

\textbf{\underline{Exemplo 2}}:\\

\begin{center}
  $ L(G) = \{ a^{n}b^{m}c^{m}d^{2n} \ | \ n \geqslant \ 0,\  m  \ > \ 0 \}$ \\
  \vspace{1cm}
  {\color{red} $ G = ${\Huge ?} } \newline
  \makebox[\linewidth]{\rule{\paperwidth/2}{0.4pt}}
\end{center}
{\color{blue}
\begin{flushleft}
    $\begin{array}{l c l }
      G & = & (V, \Sigma, P, S) \\
      V & = & \{S, A\} \\
      \Sigma & = & \{a, b, c, d\} \\
    \end{array}$  
  \[
    P:\left\{
      \begin{array}{l c l}
        S & \rightarrow & \ \texttt{a}S\texttt{dd} \quad | \quad A \\
	A & \rightarrow & \ \texttt{b}A\texttt{c} \quad | \quad \texttt{bc}  \\
      \end{array}
      \right.
   \]  
\end{flushleft}
}
\vspace{1cm}
\textbf{\underline{Exemplo 3}}:\\

\begin{center}
Uma palavra $w$ é um palíndrome se: \\
$ w = w^{r} $ \\
construa uma gramática que gere o conjunto de palíndromes sobre $\{a, b\}$.
  \makebox[\linewidth]{\rule{\paperwidth/2}{0.4pt}}
\end{center}
{\color{blue}
\begin{flushleft}
    $\begin{array}{l c l }
      G & = & (V, \Sigma, P, S) \\
      V & = & \{S, A\} \\
      \Sigma & = & \{a, b\} \\
    \end{array}$  
  \[
    P:\left\{
      \begin{array}{l c l}
        S & \rightarrow & \ \texttt{a} \quad | \quad \texttt{b} \quad | \quad \lambda \\
	S & \rightarrow & \ \texttt{a}S\texttt{a} \quad | \quad \texttt{b}S\texttt{b} \\
      \end{array}
      \right.
   \]  
\end{flushleft}
}

\vspace{1cm}
\textbf{\underline{Exemplo 4}}:\\

\begin{center}
   $G = ( \{S\}, \{a, b\}, P, S) $\\
   \[
    P:\left\{
      \begin{array}{l c l}
        S & \rightarrow & \ \texttt{a}S\texttt{b} \quad | \quad \texttt{a}S\texttt{bb} \quad | \quad \lambda \\
      \end{array}
      \right.
   \]
    {\color{red} $ L(G) = ${\Huge ?} }
    \makebox[\linewidth]{\rule{\paperwidth/2}{0.4pt}}
    \vspace{1cm}
   {\color{blue}
   $ L(G) = \{ \ a^{n}b^{m} \quad | \quad 0 \ \leqslant \ n \ \leqslant \ m \ \leqslant \ 2n \}$ \\
   }
\end{center}
\vspace{1cm}
\textbf{\underline{Exemplo 5}}:\\
   \[
      P:\left\{
      \begin{array}{l c l}
        S & \rightarrow & \ \texttt{ab}S\texttt{c}B \quad | \quad \lambda \\
	B & \rightarrow & \ \texttt{b}B  \quad | \quad \texttt{b}  \\
      \end{array}
      \right.
  \]
\begin{center}
  {\color{red} $ L(G) = ${\Huge ?} }
  \makebox[\linewidth]{\rule{\paperwidth/2}{0.4pt}}
  \vspace{1cm}
  {\color{blue} $ L(G) = \{$ enumere alguns aqui $\}$\\
   $ L(G) = \{ \ (ab)^{n}(cb^{m_n})^{n} \quad | \quad n \ \geqslant \ 0, \ m_n \ > \ 0 \}$ }
\end{center}
\vspace{1cm}
\textbf{\underline{Exemplo 6}}:\\
   \[
      G_1:\left\{
      \begin{array}{l c l}
        S & \rightarrow & \ AB \\
        A & \rightarrow & \ \texttt{a}A \quad | \quad \texttt{a}  \\
	B & \rightarrow & \ \texttt{b}B \quad | \quad \lambda  \\
      \end{array}
      \right.\] \[
      G_2:\left\{
      \begin{array}{l c l}
        S & \rightarrow & \ \texttt{a}S \quad | \quad \texttt{a}B \\
	B & \rightarrow & \ \texttt{b}B \quad | \quad \lambda  \\
      \end{array}
      \right.
  \]
\begin{center}
  {\color{red} $ L(G_1) = ${\Huge ?}\\
               $ L(G_2) = ${\Huge ?} }
  \makebox[\linewidth]{\rule{\paperwidth/2}{0.4pt}}
  \vspace{1cm}
  {\color{blue} $ L(G_1) = L(G_2) = \{ \ a^{+}b^{*} \}$ }
\end{center}

\vspace{1cm}
\textbf{\underline{Exemplo 7}}:\\

\begin{center}
$L$ é definida pela ER: \\
$ a^{*}ba^{*}ba^{*} $ \\ (palavras que contêm exatamente dois b's) \\
dar duas gramáticas $G_1$ e $G_1$ que gerem $L$.\\
  \makebox[\linewidth]{\rule{\paperwidth/2}{0.4pt}}
\end{center}
{\color{blue}
\begin{flushleft}
   \[
      G_1:\left\{
      \begin{array}{l c l}
        S & \rightarrow & \ AbAbA \\
        A & \rightarrow & \ \texttt{a}A \quad | \quad \lambda  \\
      \end{array}
      \right.\] \[
      G_2:\left\{
      \begin{array}{l c l}
        S & \rightarrow & \ \texttt{a}S \quad | \quad \texttt{b}A \\
	A & \rightarrow & \ \texttt{a}A \quad | \quad \texttt{b}C  \\
	C & \rightarrow & \ \texttt{a}C \quad | \quad \lambda  \\
      \end{array}
      \right.
  \]
\end{flushleft}
}

\vspace{1cm}
\textbf{\underline{Exemplo 8}}:\\

\begin{center}
modificar o exemplo 7 para gerar palavras de {\color{red}\underline{pelo menos}} dois b's) \\
  \makebox[\linewidth]{\rule{\paperwidth/2}{0.4pt}}
\end{center}
{\color{blue}
\begin{flushleft}
   \[
      G_1:\left\{
      \begin{array}{l c l}
        S & \rightarrow & \ AbAbA \\
        A & \rightarrow & \ \texttt{a}A \quad | \quad {\color{red}\underline{ \texttt{b}A}} \quad | \quad \lambda  \\
      \end{array}
      \right.\] \[
      G_2:\left\{
      \begin{array}{l c l}
        S & \rightarrow & \ \texttt{a}S \quad | \quad \texttt{b}A \\
	A & \rightarrow & \ \texttt{a}A \quad | \quad \texttt{b}C  \\
	C & \rightarrow & \ \texttt{a}C \quad | \quad {\color{red}\underline{ \texttt{b}C}} \quad | \quad \lambda  \\
      \end{array}
      \right.
  \]
\end{flushleft}
}

\vspace{1cm}
\textbf{\underline{Exemplo 9}}:\\

\begin{center}
Gerar a linguagem de palavras de tamanho par sobre $\{a, b\}$\\
  \makebox[\linewidth]{\rule{\paperwidth/2}{0.4pt}}
\end{center}
{\color{blue}
\begin{flushleft}
   \[
      G:\left\{
      \begin{array}{l c l}
        S & \rightarrow & \ \texttt{a}O  \quad | \quad \texttt{b}O\quad | \quad \lambda  \\
        O & \rightarrow & \ \texttt{a}S \quad | \quad \texttt{b}S \\
      \end{array}
      \right.
  \]
\end{flushleft}
}


\vspace{1cm}
\textbf{\underline{Exemplo 10}}:\\

\begin{center}
$L$ é a linguagem sobre $\{a, b\}$ com um $n^{o}$ par de b's\\
  \makebox[\linewidth]{\rule{\paperwidth/2}{0.4pt}}
\end{center}
{\color{blue}
\begin{flushleft}
   \[
      P:\left\{
      \begin{array}{l c l}
        S & \rightarrow & \ \texttt{a}S  \quad | \quad \texttt{b}I\quad | \quad \lambda  \\
        I & \rightarrow & \ \texttt{a}I \quad | \quad \texttt{b}S \\
      \end{array}
      \right.
  \]
     \[
      P^{'}:\left\{
      \begin{array}{l c l}
        S & \rightarrow & \ \texttt{a}S  \quad | \quad \texttt{b}I\quad | \quad \lambda  \\
        I & \rightarrow & \ \texttt{a}I \quad | \quad \texttt{b}S \quad | \quad \texttt{b}C\\
        C & \rightarrow & \ \texttt{a}C \quad | \quad \lambda \\
      \end{array}
      \right.
  \]
\end{flushleft}
}

\vspace{1cm}
\textbf{\underline{Exemplo 11}}:\\

\begin{center}
$L$ é a linguagem sobre $\{a, b\}$ com um $n^{o}$ par de a's e um $n^{o}$ par de b's\\
  \makebox[\linewidth]{\rule{\paperwidth/2}{0.4pt}}
\end{center}
{\color{blue}
\begin{flushleft}
   \[
      P:\left\{
      \begin{array}{l c l}
        S & \rightarrow & \ \texttt{a}B \quad | \quad \texttt{b}A\quad | \quad \lambda  \\
        A & \rightarrow & \ \texttt{a}C \quad | \quad \texttt{b}S \\
        B & \rightarrow & \ \texttt{a}S \quad | \quad \texttt{b}C \\
        C & \rightarrow & \ \texttt{a}A \quad | \quad \texttt{b}B \\
      \end{array}
      \right.
  \]
\begin{center}
\begin{tabular}{l | c c}
  & a's & b's\\
  \hline
S & par & par\\
A & par & Impar\\
B & impar & par\\
C & impar & impar
\end{tabular}
\end{center}

\end{flushleft}
}

\vspace{1cm}
\textbf{\underline{Exemplo 12}}:\\

\begin{center}
Linguagem sobre $\{a, b, c\}$ de todas as palavras que não contém a subpalavra \texttt{abc}\\
  \makebox[\linewidth]{\rule{\paperwidth/2}{0.4pt}}
\end{center}
{\color{blue}
\begin{flushleft}
   \[
      P:\left\{
      \begin{array}{l c l}
        S & \rightarrow & \ \texttt{a}A \quad | \quad \texttt{b}S \quad | \quad \texttt{c}S \quad | \quad \lambda  \\
        A & \rightarrow & \ \texttt{a}A \quad | \quad \texttt{b}B \quad | \quad \texttt{c}S \quad | \quad \lambda  \\
        B & \rightarrow & \ \texttt{a}A \quad | \quad \texttt{b}S \quad | \quad \lambda  \\
      \end{array}
      \right.
  \]
\begin{center}
\begin{itemize}
 \item S - Estaca Zero
 \item A - Já colocou um \texttt{a}
 \item B - Já colocou um \texttt{ab}
\end{itemize}

\end{center}

\end{flushleft}
}

